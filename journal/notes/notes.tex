\documentclass[letterpaper,english,12pt]{article}

\begin{document}

\textbf{5/30/2014}
\begin{itemize}
	\item SAGA
		\begin{itemize}
			\item standardized API for developing distributed applications that can run on grid and cloud infrastructure.
			\item A light-weight access layer for distributed computing infrastructure.
			\item Emphasis on job handling, monitoring, file transfer/management
			\item Goals:
				\begin{itemize}
					\item Uniform access to distributed computing infrastructures and middleware
					\item Stable programming interface for distributed applications/frameworks/tool development
					\item Ease of use
					\item Simple user-space deployment in heterogeneous distributed computing environments
				\end{itemize}
			\item Implements flexible \underline{adaptor} architecture.
			\item Adaptor: dynamically loadable modules that interface the API with different middleware systems and services.
		\end{itemize}
	\item Job Submission Systems: SSH and GSISSH, PBS and Torque, Sun Grid Engine
	\item File/Data Management: HTTP/HTTPS SFTP/GSIFTP
	\item Resource Management/Clouds: EC2(libcloud)
	\item PanDA Project / Brookhaven National Laboratory
		\begin{itemize}
			\item BNL uses RADICAL-SAGA to extend PanDA (the workload management system for the ATLAS project) to US HPC resources.
			\item It is an advanced scheduling and analysis tool
			\item Short for Production and Distributed Analysis.
			\item Manages ATLAS's data tasks from CERN server.
			\item Workflow is 1.8 million computing jobs/day distributed among about 100 computing centers worldwide.
			\item Manages 150 petabytes of data (about 75 million hours of HD video).
			\item Scheduling is an issue. Not working at full capacity if you're not using every node in your super-computing system.
			\item Grid System
				\begin{itemize}
					\item Uses a Tier system. Data is passed from tier to tier for analysis
					\item Tier-0: LHC @ CERN. Gets the raw data, then passes to Tier-1
					\item Tier-1: ten locations which recieve the data from Tier-0. BNL is in Tier-1. Connected directly to Tier-0 using a dedicated high-performance optical network path.
					\item Tier-2: Computing facilities which provide data storage and processing capacities for in-depth user analysis and simulation. Ex.: other facilities in the US. I think RADICAL is here.
				\end{itemize}
			\item Grid Infrastructure
				\begin{itemize}
					\item Consists of three key components: "fabric", "applications", "middleware"
					\item Fabric: Hardware. Processor farms with thousands of computing nodes, disk and tape storage, networking
					\item Applications: Software that scientists use to interpret the data
					\item Middleware: links the Fabric and the Applications. I believe SAGA fits here.
				\end{itemize}
		\end{itemize}
	\item Physics
		\begin{itemize}
			\item Field associated with Higgs necessary for other particles to have mass
			\item Higgs boson is very massive and decays almost instantly.
			\item LHC creates 800 million collisison between protons per second, yet it only creates a Higgs boson only once every 1-2 hours.
		\end{itemize}
\end{itemize}

\textbf{6/22/14}
\begin{itemize}
	\item Distributed Computing
		\begin{itemize}
			\item Distributed system: network of autonomous computers that communicate with each other in order to achieve a goal.
			\item Computers in the distributed system are independent and do not physically share memory or processes
			\item Communite through messages.
		\end{itemize}
	\item Client/Server Systems
		\begin{itemize}
			\item Single server that provides a service, and multiple clients that communicate with the server to consumer its products.
			\item Client does not need to know how the service is provided or how the data is calculated.
			\item Server does not need to know how the data is going to be used.
			\item \textbf{Drawback: } If the server goes down, the entire system stops working.
			\item \textbf{Drawback: } Resources become scarce if there are too many clients. Cliients increase the demand on the system without contributing any computing resources
			\item \textbf{Drawback: } Client-server systems cannot shrink and grow with changing demand.
		\end{itemize}
	\item Peer-to-Peer Systems
		\begin{itemize}
			\item Labor is divided among all components of the system
			\item All the computers send and receive data, and they all contribute some processing power and memory.
			\item Peers need to communicate with each other reliably. Need organized network structure.
			\item Data transfer and storage are most common applications.
			\item Data transfer: each computer contributes to send data over the network. If the destination computer is in a particular computer's neighborhood, that computer helps send data along. 
			\item Data storage: data set may be too large to store on one computer. Portions are stored on each computer in the system.
		\end{itemize}
	\item Modularity
		\begin{itemize}
			\item Components of a system should be black boxes with respect to each other.
			\item Easy to understand, change, and expand.
			\item Defective components can be easily swapped out
			\item Bugs/malfunctions are easy to localize
		\end{itemize}
	\item Message Passing
		\begin{itemize}
			\item Message has three parts: \textbf{sender}, \textbf{recipient}, \textbf{content}.
			\item Sender and receiver must be explicitly encoded in the message.
			\item Message content can be complex data structures, but they are all sent as 1s and 0s.
			\item Message protocol: set of rules for encoding and decoding messages.
			\item All components in distributed system must understand the protocol in order to communicate with each other.
		\end{itemize}
	\item Correctness in Parallel Computation
		\begin{itemize}
			\item Two criteria: Outcome should always be the same. Outcome should be the same as if the code was executed sequentially.
			\item Critical section: sections of code that need to be executed as if they were a single instruction, but are actually made up of smaller statements.
			\item Atomicity: quality that describes instructions that cannot be broken into smaller units or interrupted because of the design of the processor. 
			\item Serialization: processes temporarily act as if they were being executed in serial.
			\item Synchronization: uses mutual exclusion and conditional synchronization
				\begin{itemize}
					\item Mutual exclusion: processes taking turns to access a variable
					\item Conditional synchronization: processes wait until a condition is satisfied before continuing.
				\end{itemize}
		\end{itemize}
	\item Protecting Shared State: Locks and Semaphores
		\begin{itemize}
			\item Locks/Mutexes: shared objects that are commonly used to signal that shared state is being read or modified.
			\item Python uses \textbf{acquire()} and \textbf{release()}
			\item When a lock is acquired by a process, any other process that tries to perform the \textbf{acquire()} action will automatically be made to wait until the lock becomes free.
			\item Semaphores: signals used to protect access to limited resources. Only a certain number of processes are allowed to access the shared data.

			\item Condition variables:
				\begin{itemize}
					\item Objects that act as signals that a condition has been satisfied.
					\item Processes that need a condition to be satisfied can make themselves wait on a condition variable until some other process modifies it to tell them to proceed.
					\item Python uses \textbf{condition.wait()} to wait on a CV.
					\item \textbf{condition.notify()} wakes up one process, and \textbf{condition.notifyAll()} wakes up all waiting processes.
				\end{itemize}
		\end{itemize}
	\item Deadlock
		\begin{itemize}
			\item Two or more processes are stuck, waiting for each other to finish.
			\item \textbf{Circular Wait: } No process can continue because it is waiting for other processes that are waiting for it to complete.
			\item \textbf{No preemption: } One process cannot just yank a shared variable from another process that is using it
			\item \textbf{Hold and wait: }
			\item \textbf{Mutual Exclusion: }
			\item How to prevent deadlock: lock the mutexes in the same order.
		\end{itemize}
\end{itemize}
			



\begin{thebibliography}{1}
	\bibitem{cit1} Williams, Leo. "World's Most Powerful Accelerator Comes to Titan with a High-Tech Scheduler." {\em World's Most Powerful Accelerator Comes to Titan with a High-Tech Scheduler}. Brookhaven National Laboratory, 7 May 2014. Web. 31 May 2014.

	\bibitem{cit2} "Computing Support for ATLAS." {\em Computing}. Brookhaven National Laboratory, n.d. Web. 31 May 2014.
	\bibitem{cit3} "Chapter 4: Distributed and Parallel Computing." \em{Chapter 4: Distributed and Prallel Computing."}. University of California, Berkeley, n.d. Web. 22 June 2014.
\end{thebibliography}

\end{document}
